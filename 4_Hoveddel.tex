\section{Hoveddel}
Her skal det være en presentasjon av gjennomgående vansklige temaer for gruppen.

Gruppen har valgt å trekke frem tre situasjoner fra arbeidsperioden som viser ulike aspekter og utfordringer ved teamarbeidet. Den første situasjonen handler om gruppens utfordringer knyttet til dominans. En utfordring som har fulgt gruppen hele veien, og hvor tiltakene har vært effektive. Den andre situasjonen var valget av problemstilling. Dette var gruppens hovedkonflikt og viser mye spennende gruppedynamikk. Den siste situasjonen tar for seg det samholdet i gruppen. Helt fra starten har dette gode samholdet hjulpet gruppen gjennom konflikter og bidratt til et godt sosialt miljø.

\subsection{Fasiliterings øvelser}
Presentasjon av fasiliteringsøvelser.

\subsubsection{Bli kjent øvelser} %Hva er formålet? En kort beskrivelse av øvelsene, eller en mer omfattende beskrivelse om hvordan denne øvelsen var for gruppen+refleksjoner?
I oppstartfasen ble det gjennomført flere øvelser for å skape et grunnlag for samhold til landsbyen og bekjentskap mellom deltakerne. En gruppe er definert som en samling med enkelt individer som har et samspill seg imellom\footnote[1]{Johnson \& Johnson, 2013}. Slike øvelser legger til rette for at landsbymedlemmene opplever et godt samspill allerede i etableringsfasen. Etter fellesøvelsene ble landsbymedlemmene delt inn i fire grupper på fire til fem personer. 

Alternativt kan en gruppe defineres som; en sosial enhet bestående av to eller flere som føler tilhørighet\footnote[1]{Johnson \& Johnson, 2013}. Derfor var det viktig for gruppen å skape et sosialt miljø hvor medlemmene trygt kan bidra med innspill til diskusjonen. Karoline beskrev det godt i sin personlige refleksjonslogg: \textit{"Det virker som om alle i gruppen er interessert i å lytte til hverandre"}. Å føle seg sett og hørt i en gruppe er viktig for å skape engasjement og tilhørighet. Slik motiveres medlemmene til økt deltagelse og skaper en grobunn for et godt samarbeid.
%unødvendig?
De første landsbydagene var det usikkerhet rundt det manglende medlemmet, Mattis. Gruppen følte ikke dette påvirket samspillet i større grad, men det ble ytret bekymring under grupperefleksjonen på dag 2. Iselin mente det var litt synd hvis Mattis plutselig skulle møte opp midt i semesteret. 
\newline

Det var Mads tok initiativ på fritiden til å kontakte Mattis for å avklare situasjonen og fikk bekreftet at han ikke skulle ta faget. Ole Vebjørn skrev i refleksjonsloggen uken etter: \textit{"Mattis skal ikke ta faget. Det er synd for han kunne ha bidratt med mye verdifulle kompetanse"}. 

\subsubsection{SITRA}

\subsubsection{Spørreundersøkelse}

Text.

\subsection{Situasjon - Dominans} % Skrive en innledning om utvalg av disse temaene/situasjonene, feks at det har vært disse temaene som har vært gjennomgående for oss som gruppe..?
Allerede fra første landsbydag ble gruppen bevisstgjort på at det var fire sterke personligheter i gruppen. Den første landsbydagen resulterte i tydelige tegn på at samtlige medlemmer i gruppen til enhver tid hadde mye de ønsket å formidle til de andre medlemmene. Dette har resultert i en lang rekke opphetede diskusjoner gjennom semesteret, med de fordeler og ulemper det måtte bringe.  


Den fjerde landsbydagen ble det tydeliggjort for gruppen at det var skjev taletid mellom medlemmene. En av fasilitatorene hadde tegnet et skjema som viste hvilke medlemmer som tok ordet hyppigere enn andre, og til hvem medlemmene henvendte seg til mens de snakket. Det kom tydelig frem av skjemaet at Mads tar ordet hyppigere enn de andre medlemmene. Det viste seg at samtalen ble observert over en tidsperiode på 10 minutter hvor Mads skulle forklare de andre medlemmene om et tema han hadde god kunnskaper om. Gruppen valgte likevel å diskutere hvorvidt dette skjemaet var representativt for gruppens samtaledynamikk. 
\subsubsection{Individuell oppfatning}
I diskusjonen tok Mads selv ordet og påpekte at han er klar over at han kan virke dominant i en gruppe med nye mennesker. Mads fortalte at han er vandt til å ta en lederrolle i gruppesammenhenger, og at han er svært bevisst på at hans oppførsel kan virke dominant ovenfor andre. De øvrige gruppemedlemmene lyttet til Mads før de ytret sitt synspunkt i diskusjonen.
%Her kan alle selv skrive hva de synes om hendelsen.
Iselin fortalte gruppen at hun satt pris på Mads sitt engasjement og entusiasme. Ole Vebjørn påpekte at skjemaet viste at samtalen Mads første var rettet til midten av bordet, hvilket betyr at Mads henvendte seg til alle når han snakket, noe Ole Vebjørn mente var positivt. Karoline fortalte at hun forsåvidt var enig med Iselin og Ole Vebjørn, men tok opp at hun fryktet at dominans i gruppen kunne føre til at noen medlemmer ikke følte de fikk tilstrekkelig taletid, eller at de ikke ville tørre å ta ordet i en diskusjon. 
\subsubsection{Gruppens refleksjoner}
Iselin kom da med forslaget om et tiltak for å hindre dominans i gruppen. Tiltaket som ble vedtatt 
\subsubsection{Forbedringstiltak}

\subsubsection{Resultat og oppfølging}




\subsection{Situasjon - Problemstillings diskusjonen}
Kort utenforstående beskrivelse av situasjonen.
\subsubsection{Individuell oppfatning}
Individuelle oppfatninger/opplevelse (dette kan de skrive selv).
\subsubsection{Grupperefleksjon}
Gruppens refleksjoner rundt situasjonen.
\subsubsection{Forbedringstiltak}
Aksjoner og tiltak som ble gjennomført.
\subsubsection{Resultat og oppfølging}
Re-evaluering av aksjoner og tiltak.

\subsection{Situasjon - Samhold (positiv)}
Kort utenforstående beskrivelse av situasjonen.
\subsubsection{Individuell oppfatning}
Individuelle oppfatninger/opplevelse (dette kan de skrive selv).
\subsubsection{Grupperefleksjon}
Gruppens refleksjoner rundt situasjonen.
\subsubsection{Forbedringstiltak}
Aksjoner og tiltak som ble gjennomført.
\subsubsection{Resultat og oppfølging}
Re-evaluering av aksjoner og tiltak.